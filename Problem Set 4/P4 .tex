%---------change this every homework
\def\yourname{Yingzhu (Jacqueline) Zhang}
% -----------------------------------------------------
\def\homework{0} % 0 for solution, 1 for problem-set only
\def\duedate{mon sep 24, 2014}
\def\duelocation{}
\def\hnumber{4}
\def\prof{Prof. Lewis}
\def\course{CS 303 Algorithms}%-------------------------------------
%-------------------------------------

\documentclass[11pt]{article}
\usepackage[colorlinks,urlcolor=blue]{hyperref}
\usepackage[osf]{mathpazo}
\usepackage{amsmath,amsfonts,graphicx}
\usepackage{latexsym}
\usepackage[top=1in,bottom=1.4in,left=1.5in,right=1.5in,centering]{geometry}
\usepackage{color}
\usepackage{amssymb}
\usepackage{enumerate}
\definecolor{mdb}{rgb}{0.3,0.02,0.02} 
\definecolor{cit}{rgb}{0.05,0.2,0.45} 
\pagestyle{myheadings}
\markboth{\yourname}{\yourname}
\usepackage{clrscode}


\newenvironment{proof}{\par\noindent{\it Proof.}\hspace*{1em}}{$\Box$\bigskip}
\newcommand{\qed}{$\Box$}
\newcommand{\alg}[1]{\mathsf{#1}}
\newcommand{\handout}{
   \renewcommand{\thepage}{H\hnumber-\arabic{page}}
   \noindent
   \begin{center}
      \vbox{
    \hbox to \columnwidth {\sc{\course} --- \prof \hfill}
    \vspace{-2mm}
    \hbox to \columnwidth {\sc due \MakeLowercase{\duedate} \duelocation\hfill {\Huge\color{mdb}H\hnumber.}}
	\vspace{15pt}
	{\Huge\yourname}
      }
   \end{center}
   \vspace*{2mm}
}
\newcommand{\solution}[1]{\medskip\noindent\textbf{Solution:}#1}
\newcommand{\bit}[1]{\{0,1\}^{ #1 }}
%\dontprintsemicolon
%\linesnumbered
\newtheorem{problem}{\sc\color{cit}problem}
\newtheorem{practice}{\sc\color{cit}practice}
\newtheorem{lemma}{Lemma}
\newtheorem{definition}{Definition}

\begin{document}
\thispagestyle{empty}
\handout

Students I have consulte with for this assignment:\\
1. William Bench\\
2. Nathaniel Marshall\\
3. Jacob Shor\\
4. William Theuer\\
5. Maurice Hayward\\
6. Stephanie Bramlett\\
7. Diego Lanao\\
8. Meade Nelson\\
9. Kalyn Horn

\begin{enumerate}

\item 2.3.4.
\begin{solution}\\
$Example 1:$ A B C D E F G H I J\\
$Example 2:$ K L M N O P Q R S T\\
$Example 3:$ 1 2 3 4 5 6 7 8 9 10\\
$Example 4:$ 11 13 15 17 19 21 23 25 27 29\\
$Example 5:$ 10 9 8 7 6 5 4 3 2 1\\
$Example 6:$ J I H G F E D C B A
\end{solution}

\item 2.3.10.
\begin{solution}\\
Given: $N = 10^6, .1N^2 = 10^11 $\\
Also, according to Proof of Proposition L, the standard deviation for quicksort is $.65N$\\
And we know that the mean of quicksort compares is $2NlnN$\\
$\therefore$ $k = \frac {.1N^2 - 2NlnN}{.65N}$\\
$\therefore$ Probability $P = \frac{1}{\frac{10^11-2(10^6)ln(10^6)}{.65(10^6)}} \approx 4.2273 \times 10^{-11}$
\end{solution}

\item 2.3.13.
\begin{solution}\\
Because the recursive call the quicksort performs is when it performs $partition()$;\\
Also because a stack contains pertinent information for each recursive call. When a procedure is invoked, its information is pushed onto the stack; when it terminates, its information is popped\\
$\therefore$\\
best case: $lgN$ ($partition ()$ splits the array exactly in half)\\
worst case: $N-1$ ($partition ()$ splits the array as unevenlly as possible)\\
best case: $lgN$ ($partition ()$ falls in the middle on average)\\

\end{solution}

\item Give the heap that results when the keys P O S S U M L O A F are inserted in that order into an initially empty max-oriented heap.\\
\begin{solution}\\
See attached sheets of paper for the drawing.
\end{solution}

\item 2.4.7.
\begin{solution}\\
when $k = 2$, $k$th largest position can appear in the the leaves of height 2; cannot appear in any other leaves\\
when $k = 3$, $k$th largest position can appear in the leaves of heights 2 and 3; cannot appear in any other leaves\\
when $k = 4$, $k$th largest position can appear in the leaves of heights 2, 3 and 4; cannot appear in any other leaves\\
See attached sheets of paper for an illustration of the height of the heap.
\end{solution}

\item Give an example that shows that quicksort is not stable.\\
\begin{solution}\\
Suppose we have an array containing $F\ O^{1}\ O^{2}\ D \ O^{3}\ O^{4}$\\
Using quicksort, our sorted result would be:\\
$F D\ O^{2}\ O^{1} D \ O^{3}\ O^{4}$\\
$D F\ O^{4}\ O^{1} D \ O^{3}\ O^{2}$\\
However, if we were to use a stable sort method, such as mergesort, the result would be:\\
$D F\ O^{1}\ O^{2} D \ O^{3}\ O^{4}$\\
$\therefore$ We can see here that quicksort is not stable because it does not account for the different values associated with the same entry. 
\end{solution}

\item 3.3.1.
\begin{solution}\\
See attached sheets of paper for the drawing.
\end{solution}

\item 3.3.3.
\begin{solution}\\
In order to find the insertion order, we can first sort the key in ascending order, and we get\\
$A C E H M R S X$\\
Then, we draw a 2-3 tree whose height is one (see attached paper);\\
Then, we try to draw out the 2-3 tree which results this array of keys step-by-step (see attached paper) and collect the insertion order;\\
Then we get the insertion order as:\\
$R A X H S E C M$\\
\end{solution}

\break

\item 3.3.4.
\begin{solution}\\
Given a list of N objects, when we contruct for a tree that is all 2-nodes, this tree of height $h$ must have at most $2^d+1$ leaves\\
Similarly, when we constrct for a tree that is all 3-nodes, this tree of height $h$ must have at most $3^d+1$ leaves\\
$\therefore$ We have an equation:\\
$N \leq 2^d$ or $N \leq 3^d$, which is $lgN \leq h$ or $log_3N \leq h$ \\
$\therefore$ It is true that the height of a 2-3 tree with $N$ keys is between $\sim$ $\lfloor$ $log_3N$ $\rfloor$ (for a tree that is all 3-nodes) and $\sim$ $\lfloor$ $lgN$ $\rfloor$ (for a tree that is all 2-nodes).
\end{solution}


\end{enumerate}


\end{document}