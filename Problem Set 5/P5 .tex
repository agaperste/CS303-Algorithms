%---------change this every homework
\def\yourname{Yingzhu (Jacqueline) Zhang}
% -----------------------------------------------------
\def\homework{0} % 0 for solution, 1 for problem-set only
\def\duedate{wed oct 1, 2014}
\def\duelocation{}
\def\hnumber{5}
\def\prof{Prof. Lewis}
\def\course{CS 303 Algorithms}%-------------------------------------
%-------------------------------------

\documentclass[11pt]{article}
\usepackage[colorlinks,urlcolor=blue]{hyperref}
\usepackage[osf]{mathpazo}
\usepackage{amsmath,amsfonts,graphicx}
\usepackage{latexsym}
\usepackage[top=1in,bottom=1.4in,left=1.5in,right=1.5in,centering]{geometry}
\usepackage{color}
\usepackage{amssymb}
\usepackage{enumerate}
\definecolor{mdb}{rgb}{0.3,0.02,0.02} 
\definecolor{cit}{rgb}{0.05,0.2,0.45} 
\pagestyle{myheadings}
\markboth{\yourname}{\yourname}
\usepackage{clrscode}


\newenvironment{proof}{\par\noindent{\it Proof.}\hspace*{1em}}{$\Box$\bigskip}
\newcommand{\qed}{$\Box$}
\newcommand{\alg}[1]{\mathsf{#1}}
\newcommand{\handout}{
   \renewcommand{\thepage}{H\hnumber-\arabic{page}}
   \noindent
   \begin{center}
      \vbox{
    \hbox to \columnwidth {\sc{\course} --- \prof \hfill}
    \vspace{-2mm}
    \hbox to \columnwidth {\sc due \MakeLowercase{\duedate} \duelocation\hfill {\Huge\color{mdb}H\hnumber.}}
	\vspace{15pt}
	{\Huge\yourname}
      }
   \end{center}
   \vspace*{2mm}
}
\newcommand{\solution}[1]{\medskip\noindent\textbf{Solution:}#1}
\newcommand{\bit}[1]{\{0,1\}^{ #1 }}
%\dontprintsemicolon
%\linesnumbered
\newtheorem{problem}{\sc\color{cit}problem}
\newtheorem{practice}{\sc\color{cit}practice}
\newtheorem{lemma}{Lemma}
\newtheorem{definition}{Definition}

\begin{document}
\thispagestyle{empty}
\handout

Students I have consulte with for this assignment:\\
1. Diego Lanao\\
2. Liam Bench\\
3. Jake Shor\\
4. Nate Marshalls\\
5. Maurice Hayward\\
6. Tyler Reid

\begin{enumerate}

\item 3.3.10
\begin{solution}\\
(See attached paper.)
\end{solution}

\item 3.3.11
\begin{solution}\\
(See attached paper.)
\end{solution}

\item 3.4.1

\begin{solution}\\
\begin{quote}
\begin{tabular}{|c|c|c|c|c|c|c|c|c|c|c|c}
\bf key&E&A&S&Y&Q&U&T&I&O&N\\
\hline
value&5&1&19&25&17&21&20&9&15&14\\
\hline
\bf hash&0&1&4&0&2&1&0&4&0&4
\end{tabular}
\end{quote}
* We hash by using hash function $11k\%M$, where $M=5$\\
$\therefore$ The table index is:
\begin{quote}
\begin{tabular}{ccccc}
0&1&2&3&4\\
\hline
E&A&Q&&S\\
Y&U&&&I\\
T&&&&N\\
O&&&&\\
\end{tabular}
\end{quote}

\end{solution}


\item 3.4.6

\begin{solution}\\
{\bf Proposition:} For a modular hash function with prime $M$, two keys that are interger differring by $2^p$ with $p \in N$  have different hash values.\\
$Proof:$\\
We prove by contradiction.\\
Suppose the proposition above is false, which is, for a modular hash function with prime $M$, two keys that are interger differring by $2^p$ with $p \in N$  have the same hash values\\
Then,

\begin{quote}
$k\%M=(k-2^p)\%M$\\
For key $a$, we have:\\
$k=aM+r, a \in \mathbb{N}$\\
For key $b$, we have:\\
$k-2^p=bM+r, b \in \mathbb{N}$\\
$aM+r =bM+r+2^p$\\
$(a-b)M=2^P$
\end{quote}

If $(a-b)M=2^P$, both sides would have to contain $M$. However, this is not possible for the right side of the equation given that $M \not= 2$. This is because $2^p$ only gives prime factorization with values of 2 without $M$.\\
Following that, the contradiction of the proposition is false.\\
$\therefore$ The proposition "For a modular hash function with prime $M$, two keys that are interger differring by $2^p$ with $p \in N$  have different hash values" is $true$. 

\end{solution}


\item 3.4.7

\begin{solution}\\
{\bf Proposition:} Implementing modular hashing for integer keys with the code $(a \times k) \% M$, where $a$ is an arbitrary fixed prime, will not mix up the bits sufficiently well that we can use nonprime $M$.\\
$Proof:$\\
We prove this by example. \\
We first randomly select some number, $25, 37, 48, 91, 145;$ and sets $a$ to be a prime number 2, $M$ to be 4 (or $2^2$)\\
Then, the table index would be:

\begin{quote}
\begin{tabular}{cccc}
0&1&2&3\\
\hline
48&&25&\\
&&37&\\
&&91&\\
&&145&\\
\end{tabular}
\end{quote}

$\therefore$ As shown above, implementing modular hashing for integer keys with the code $(a \times k) \% M$, where $a$ is an arbitrary fixed prime, will not mix up the bits sufficiently well that we can use nonprime $M$.
\end{solution}


\item 3.4.13

\begin{solution}\\
Scenario $a$ leads to expected linear running time for a random search hit in a linear-probing hash table.\\
$Explanation:$\\
If all keys hash to the same index, then the $i$th key inserted requires $i$ times of loopups to be found.\\
Because the probability of looking up $i$th key is $1/n$, as it is random,\\
$\therefore$ All keys hash to the same index results into an expected linear running time. 
\end{solution}


\item 3.3.15

\begin{solution}\\
\begin{quote}
$\frac{N(N+1)}{2}$
\end{quote}
\end{solution}

$Proof:$\\
To insert $N$ keys into an initially empty table using linear probing with array resizing, we are increasing the size of the array. For instance, we start with $i[0]$ and add to $i[1]$. As we keep incrementing the index, we are also incrementing the number of compares required in such a fashion:\\
\begin{tabular}{c|c|c|c|c|}
item&1&1&1&...\\
\hline
index&0&1&2&...\\
\end{tabular}, 
which results $1, 2, 3, ..., \frac{N(N+1)}{2}$ compares.


\end{enumerate}


\end{document}