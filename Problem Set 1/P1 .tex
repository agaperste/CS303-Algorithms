%---------change this every homework
\def\yourname{Yingzhu (Jacqueline) Zhang}
% -----------------------------------------------------
\def\homework{0} % 0 for solution, 1 for problem-set only
\def\duedate{mon sep 8, 2014}
\def\duelocation{}
\def\hnumber{1}
\def\prof{Prof. Lewis}
\def\course{CS 303 Algorithms}%-------------------------------------
%-------------------------------------

\documentclass[11pt]{article}
\usepackage[colorlinks,urlcolor=blue]{hyperref}
\usepackage[osf]{mathpazo}
\usepackage{amsmath,amsfonts,graphicx}
\usepackage{latexsym}
\usepackage[top=1in,bottom=1.4in,left=1.5in,right=1.5in,centering]{geometry}
\usepackage{color}
\usepackage{amssymb}
\usepackage{enumerate}
\definecolor{mdb}{rgb}{0.3,0.02,0.02} 
\definecolor{cit}{rgb}{0.05,0.2,0.45} 
\pagestyle{myheadings}
\markboth{\yourname}{\yourname}
\usepackage{clrscode}


\newenvironment{proof}{\par\noindent{\it Proof.}\hspace*{1em}}{$\Box$\bigskip}
\newcommand{\qed}{$\Box$}
\newcommand{\alg}[1]{\mathsf{#1}}
\newcommand{\handout}{
   \renewcommand{\thepage}{H\hnumber-\arabic{page}}
   \noindent
   \begin{center}
      \vbox{
    \hbox to \columnwidth {\sc{\course} --- \prof \hfill}
    \vspace{-2mm}
    \hbox to \columnwidth {\sc due \MakeLowercase{\duedate} \duelocation\hfill {\Huge\color{mdb}H\hnumber.}}
	\vspace{15pt}
	{\Huge\yourname}
      }
   \end{center}
   \vspace*{2mm}
}
\newcommand{\solution}[1]{\medskip\noindent\textbf{Solution:}#1}
\newcommand{\bit}[1]{\{0,1\}^{ #1 }}
%\dontprintsemicolon
%\linesnumbered
\newtheorem{problem}{\sc\color{cit}problem}
\newtheorem{practice}{\sc\color{cit}practice}
\newtheorem{lemma}{Lemma}
\newtheorem{definition}{Definition}

\begin{document}
\thispagestyle{empty}
\handout

Students I have consulte with for this assignment:\\
1. William Bench\\
2. Nathanel Marshall\\
3. Yussre ElBardicy\\
4. William Theuer\\
5. Maurice Hayward\\
6. Stephanie Bramlett\\
7. Diego Lanao\\

\begin{enumerate}

\item 1.4.4
\begin{solution}\\
public class twoSum\\
\{ \\
\-\ \-\ \-\ \-\ public static in count(int[] a) \\
 \-\ \-\ \-\ \-\ \{\\
		\-\ \-\ \-\ \-\ int N = a.length;\\
		\-\ \-\ \-\ \-\ int cnt = 0;\\
		\\
		\-\ \-\ \-\ \-\ for (int i = 0; i <N; i ++)\\
			\-\ \-\ \-\ \-\ \-\ \-\ \-\ \-\ for (int j = i+1; j < N; j++)\\
				\-\ \-\ \-\ \-\ \-\ \-\ \-\ \-\ \-\ \-\ \-\ \-\ if (a[i] + a[j] == 0)\\
		\-\ \-\ \-\ \-\  \-\ \-\ \-\ \-\ \-\ \-\ \-\ \-\ \-\ \-\ \-\ \-\ cnt ++;\\
		\\
		\-\ \-\ \-\ \-\ return cnt;\\
	\-\ \-\ \-\ \-\ \}\\
	\\
	\-\ \-\ \-\ \-\ public static void main(String[] args)\\
	\-\ \-\ \-\ \-\ \{\\
		\-\ \-\ \-\ \-\ \-\ \-\ \-\ \-\ int[] a = In.readInts(args[0]);\\
		\-\ \-\ \-\ \-\ \-\ \-\ \-\ \-\ StdOut.println(count(a))\\
	\-\ \-\ \-\ \-\ \}\\
\}\\

\begin{tabular}{cccc}
\bf statement block & \bf time in seconds & \bf frequency & \bf total time\\
\hline
D & $t_0$ & $x$(depends on input) & $t_0 x$\\
C & $t_1$ & $N^2 / 2 - N/2$& $t_1 (N^2 / 2 - N/2)$\\
B & $t_2$ & $N$ & $t_2 N$\\
A & $t_3$ & $1$ & $t_3$\\
\\
& \bf grand total & $(t_1 /2)N^2$\\
&& $(-t_1 /2 + t_2)N$\\
&& $t_3 + t_0 x$\\
& \bf tilde approximation & $ \sim (t_1 /2)N^2$ (assuming x is small)\\
& \bf order of growth &$N^2$\\
\end{tabular}



\end{solution}

\item 1.4.5
\begin{solution}

\begin{enumerate}[(a)] 
\item $\sim N$
\item $\sim 1$
\item $1+2/N + 1/N + 2/N^2 = 1+ 1/N + 2/N + 2/N^2$ \space \space $\sim 1$
\item $\sim 2N^3$
\item $ lg 2N -lg N = (lg2 + lgN) / lgN = 1/lgN + 1$ \space \space $\sim 1$
\item $lg(N^2 + 1) / lgN  = lg(1 (N^2 +1)) / lgN \\
= (lg (N^2 (1+ 1/N^2))) / lgN \\
= (2 lgN +lg(1+1/1/N^2))) / lgN \\
= 2 + ((lg(1+1/N^2) /lgN))$ \space \space \space $\sim 2$
\item $\sim N^{100} / 2^N $
\end{enumerate}

\end{solution}

\item 1.4.9
\begin{solution}\\
According to Doubling Ratio Proposition:\\
If $T(N) \sim aN^b lgN$, then $T(2N) / T(N) =$\\
$(a(2N)^b lg(2N)) / aN^b lgN$\\
$(2^b N^b lg(2N)) / N^b lgN$\\
$(2^b lg(2N)) / lgN$\\
$(2^b (lg2 lgN)) / lgN = 2^b(1)$\\
$\therefore$  \space $\sim 2^b$\\

Because the running time for problems of size $N_0= T$, and the doubling factor converges to $2^b$ then\\
According to Doubling Ratio Proposition, \\
$aN_0 ^b lgN_0 = T$, $a = T/ (N_0 ^b lg N_0)$ \\
$\therefore$ the running time of a program for a problem of size N: $\sim T (N^b lgN)/ (N_0 ^b lg N_0)$

\end{solution}

\item 1.5.1
\begin{solution}\\
Through quick-find, we get: \\
\begin{tabular}{c|cccccccccc|r}
$p$-$q$& 0&1 &2 &3 &4 &5 &6 &7 & 8 & 9 & number of times the array is accessed\\
\hline
9-0 & 0&1 &2 &3 &4 &5 &6 &7 & 8 & 0 & 2+10+1 = 13\\
3-4 & 0&1 &2 &4 &4 &5 &6 &7 & 8 & 0 & 2+10+1 = 13\\
5-8 & 0&1 &2 &4 &4 &8 &6 &7 & 8 & 0 & 2+10+1 = 13\\
7-2 & 0&1 &2 &4 &4 &8 &6 &2 & 8 & 0 & 2+10+1 = 13\\
2-1 & 0&1 &1 &4 &4 &8 &6 &1 & 8 & 0 & 2+10+1+1 = 14\\
5-7 & 0&1 &1 &4 &4 &1 &6 &1 & 1 & 0 & 2+10+1+1 = 14\\
0- 3& 4&1 &1 &4 &4 &1 &6 &1 & 1 & 4 & 2+10+1+1 = 14\\
4- 2& 4&1 &1 &1 &1 &1 &6 &1 & 1 & 1 & 2+10+1+2 = 15\\
\\
\end{tabular}
See attached sheet of paper for demonstrative diagrams.
\end{solution}

\item 1.5.2
\begin{solution}\\
Through quick-union, we get: \\
\begin{tabular}{cc|cccccccccc|r}
& $p$-$q$& 0&1 &2 &3 &4 &5 &6 &7 & 8 & 9 & number of times the array is accessed\\
\hline
1) & 9-0 & 0&1 &2 &3 &4 &5 &6 &7 & 8 & 0 & 2+1 = 3\\
2) & 3-4 & 0&1 &2 &4 &4 &5 &6 &7 & 8 & 0 & 2+1 = 3\\
3) & 5-8 & 0&1 &2 &4 &4 &8 &6 &7 & 8 & 0 & 2+1 = 3\\
4) & 7-2 & 0&1 &2 &4 &4 &8 &6 &2 & 8 & 0 & 2+1 = 3\\
5) & 2-1 & 0&1 &1 &4 &4 &8 &6 &2 & 8 & 0 & 2+1 = 3\\
6) & 5-7 & 0&1 &1 &4 &4 &2 &6 &2 & 8 & 0 & 2+2+1 = 5\\
7) & 0-3 & 4&1 &1 &4 &4 &2 &6 &2 & 8 & 0 & 1+2+1 = 4\\
8) & 4-2 & 4&1 &1 &4 &1 &2 &6 &2 & 8 & 0 & 1+2+1 = 4\\
\\
\end{tabular}
See attached sheet of paper for demonstrative diagrams.

\end{solution}

\item 1.5.3
\begin{solution}\\
Through weighted quick-union, we get: \\
\begin{tabular}{cc|cccccccccc|r}
& $p$-$q$& 0&1 &2 &3 &4 &5 &6 &7 & 8 & 9 & number of times the array is accessed\\
\hline
1) & 9-0 & 9&1 &2 &3 &4 &5 &6 &7 & 8 & 9 & 2+1+1 = 4\\
2) & 3-4 & 9&1 &2 &3 &3 &5 &6 &7 & 8 & 9 & 2+1+1 = 4\\
3) & 5-8 & 9&1 &2 &3 &3 &5 &6 &7 & 5 & 9 & 2+1+1 = 4\\
4) & 7-2 & 9&1 &7 &3 &3 &5 &6 &7 & 5 & 9 & 2+1+1 = 4\\
5) & 2-1 & 9&7 &7 &3 &3 &5 &6 &7 & 5 & 9 & 2+1+1+1 = 5\\
6) & 5-7 & 9&7 &7 &3 &3 &5 &6 &5 & 5 & 9 & 2+1+1 = 4\\
7) & 0-3 & 9&7 &7 &9 &3 &5 &6 &5 & 5 & 9 & 2+2+1+1 = 6\\
8) & 4-2 & 9&7 &7 &9 &7 &5 &6 &5 & 5 & 9 & 2+2+1+1 = 6\\
\\
\end{tabular}
See attached sheet of paper for demonstrative diagrams.
\end{solution}

\item 1.5.5
$10^9$ sites--array size/number of components),\\ 
$10^6$ input pairs, \\
$10^9$ instructions per second--array accesses persecond\\
Assume:each iteration of the inner $for$ loop requires 10 machine instruction.\\
We can find the minimus amount of time (in days) that would be required for quick-find to solve this dynamic connectivity problem by: \\
\begin{solution}\\
For each input pair:\\
$find(): 2 machine instructions$ \\
$union(): 10 * (10^9 +1) machine instructions$\\
Then, for $10^6$ pair, the total machine instructions = $2*10^6 + 10^7(10^9 +1)$\\
Also, a computer is capable of executing $24 * 60^2 * 10^9 = 8.64*10^13$ machine instructions per day\\
 $(2*10^6 + 10^7(10^9 +1)) / 8.64*10^13 \approx 115.7407$\\
$\therefore$ $\approx 116$ days 

\end{solution}

\item 1.5.9
\begin{solution}\\
According to the proposition that the depth of any node in a forest built by weighted quick-union for $N$ sites is at most $lgN$, \\
When $N= 10,$ depth = $lg10 (3<lg10<4)$\\
But according to the diagram drawn in the attachement page, the depth of this problem is 4,\\
$\therefore$ it is impossible to ge the result by running weighted quick-union 

\end{solution}

\end{enumerate}


\end{document}